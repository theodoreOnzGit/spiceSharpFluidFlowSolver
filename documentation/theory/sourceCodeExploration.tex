
%%%%%%%%%%%%  Generated using docx2latex.com  %%%%%%%%%%%%%%

%%%%%%%%%%%%  v2.0.0-beta  %%%%%%%%%%%%%%

\documentclass[12pt]{article}
\usepackage{amsmath}
\usepackage{latexsym}
\usepackage{amsfonts}
\usepackage[normalem]{ulem}
\usepackage{soul}
\usepackage{array}
\usepackage{amssymb}
\usepackage{extarrows}
\usepackage{graphicx}
\usepackage[backend=biber,
style=apa,
sorting=none,
isbn=true,
doi=true,
url=true,
]{biblatex}\addbibresource{bibliography.bib}

\usepackage{subfig}
\usepackage{wrapfig}
\usepackage{wasysym}
\usepackage{enumitem}
\usepackage{adjustbox}
\usepackage{ragged2e}
\usepackage[svgnames,table]{xcolor}
\usepackage{tikz}
\usepackage{longtable}
\usepackage{changepage}
\usepackage{setspace}
\usepackage{hhline}
\usepackage{multicol}
\usepackage{tabto}
\usepackage{float}
\usepackage{multirow}
\usepackage{makecell}
\usepackage{fancyhdr}
\usepackage[toc,page]{appendix}
\usepackage[hidelinks]{hyperref}
\usetikzlibrary{shapes.symbols,shapes.geometric,shadows,arrows.meta}
\tikzset{>={Latex[width=1.5mm,length=2mm]}}
\usepackage[utf8]{inputenc}
\usepackage[T1]{fontenc}
\TabPositions{0.5in,1.0in,1.5in,2.0in,2.5in,3.0in,3.5in,4.0in,4.5in,5.0in,5.5in,6.0in,}

\urlstyle{same}

\renewcommand{\_}{\kern-1.5pt\textunderscore\kern-1.5pt}

 %%%%%%%%%%%%  Set Depths for Sections  %%%%%%%%%%%%%%

% 1) Section
% 1.1) SubSection
% 1.1.1) SubSubSection
% 1.1.1.1) Paragraph
% 1.1.1.1.1) Subparagraph


\setcounter{tocdepth}{5}
\setcounter{secnumdepth}{5}


 %%%%%%%%%%%%  Set Depths for Nested Lists created by \begin{enumerate}  %%%%%%%%%%%%%%


\setlistdepth{9}
\renewlist{enumerate}{enumerate}{9}
		\setlist[enumerate,1]{label=\arabic*)}
		\setlist[enumerate,2]{label=\alph*)}
		\setlist[enumerate,3]{label=(\roman*)}
		\setlist[enumerate,4]{label=(\arabic*)}
		\setlist[enumerate,5]{label=(\Alph*)}
		\setlist[enumerate,6]{label=(\Roman*)}
		\setlist[enumerate,7]{label=\arabic*}
		\setlist[enumerate,8]{label=\alph*}
		\setlist[enumerate,9]{label=\roman*}

\renewlist{itemize}{itemize}{9}
		\setlist[itemize]{label=$\cdot$}
		\setlist[itemize,1]{label=\textbullet}
		\setlist[itemize,2]{label=$\circ$}
		\setlist[itemize,3]{label=$\ast$}
		\setlist[itemize,4]{label=$\dagger$}
		\setlist[itemize,5]{label=$\triangleright$}
		\setlist[itemize,6]{label=$\bigstar$}
		\setlist[itemize,7]{label=$\blacklozenge$}
		\setlist[itemize,8]{label=$\prime$}

\setlength{\topsep}{0pt}\setlength{\parskip}{8.04pt}
\setlength{\parindent}{0pt}

 %%%%%%%%%%%%  This sets linespacing (verticle gap between Lines) Default=1 %%%%%%%%%%%%%%


\renewcommand{\arraystretch}{1.3}


%%%%%%%%%%%%%%%%%%%% Document code starts here %%%%%%%%%%%%%%%%%%%%
\begin{document}

\tableofcontents

\part{Problem statement}

Pipe networks can be analogous to Electrical Circuits. Therefore, it can be tempting to use electrical flow solvers for pipe networks.

\section{Unit Comparison}

Voltage is energy per unit charge unit wise. $\frac{J}{C}$ Joules per coulomb.

Whereas Pressure can be thought of as Energy per unit volume. $\frac{J}{m^3}$ Joules per meter cubed. That would be the unit of pascals.

$$(workDone)\ Joules = (pressure)\ Pa * (volume)\ m^3$$
$$(pressure)\ Pa = \frac{(workDone)\ Joules}{(volume)\ m^3}$$

Compare this to voltage:

$$(voltage)\ Volts = \frac{(workDone)\ Joules}{(Charge)\ Coulomb}$$

Likewise for volumetric flowrate, this is

$$(flowrate)\ m^3/s = \frac{(vol)\ m^3}{(time)\ s}$$

And for current,

$$(current)\ C/s = \frac{(Charge)\ Coulomb}{(time)\ s}$$

\section{nonlinearities in flow resistance}

However, the components exhibiting flow resistance often do not obey Ohm's law.

$$electrical\ resistance\ (\Omega) = \frac{V\ (Volts)}{I\ (Ampere)}$$
$$flow\ resistance = \frac{\Delta P\ (Pa)}{\dot{V}\ (m^3)}$$


For ohm's law, the ratio V/I reduces to a constant, but flow resistance often reduces to some expression. 

If we were to use Fanning friction factor for pipe (cite perry's handbook)

$$f = \frac{\Delta P}{ (\frac{4L}{D}) \  \frac{1}{2} \rho u^2 }$$

$$\Delta P = f  (\frac{4L}{D}) \  \frac{1}{2} \rho u^2 $$

In Perry's chemical engineering handbook, the formula used for fanning's friction factor by Churchill is:

$$f = 2 \left[\\
\left( \frac{8}{Re} \right)^{12} + \\
\left( \frac{1}{A+B}\right)^{3/2} \\
\right]^{1/12} $$
 
Where:

$$A = \left[ 2.457 \ln \frac{1}{\left( \frac{1}{(7/Re)^{0.9}} + \\
0.27 \frac{\varepsilon}{D} \right)} \\
\right]\ \ ; \ \ \\
B = \left( \frac{37530}{Re} \\ 
\right)^{16} $$


$$Re = \frac{ux}{\nu} = \frac{\dot{V} x}{A_{XS} \nu}$$

Where $A_{XS}$ represents cross sectional area.
We can see that this is strongly non linear with respect to volumetric flowrate.


\part{Matrices}

$$\begin{bmatrix}
1 & 2 & 3 \\
a & b & c
\end{bmatrix}$$

 
\end{document}
